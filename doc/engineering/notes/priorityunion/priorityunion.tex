\documentclass[letterpaper,12pt]{article}
% \documentclass[a4paper,12pt]{article}
% twocolumn letterpaper 10pt 11pt twoside

% for other type sizes, 8, 9, 10, 11, 12, 14pt, 17pt, 20pt
% \documentclass[14pt]{extarticle}
% also extbook, extletter available
% \usepackage{extsizes}

\usepackage{times}
\usepackage{xspace}
%\usepackage{alltt}
\usepackage{fancyvrb}  % \begin{Verbatim}[fontsize=\small]
% or [fontsize=\footnotesize]

%\usepackage{latexsym}  % \LaTeX{} for LaTeX;  \LaTeXe{} for LaTeX2e
%\usepackage{mflogo}    % \MF{}  for METAFONT;  \MP for METAPOST
%\usepackage{url}       % \url{http://www.xrce.xerox.com/people/beesley}

%\usepackage{tipa}
%\include{ipamacros}  % my macros to allow same input for DA and IPA
%\usepackage{desalph}
%\usepackage{arabtex} % see usepackage{buck} and setcode{buck} below
%\usepackage{buck}
%\usepackage{mxedruli}

%\usepackage{epsfig}
%\usepackage{pslatex}  % make whole doc. use postscript fonts

% for more of these names, see Guide to LaTeX, p. 351
%\providecommand*{\abstractname}{}     % in case the style defines one
%\renewcommand*{\abstractname}{Transcriber notes}
%\renewcommand*{\figurename}{Figure}
%\renewcommand*{\tablename}{Table}
%\renewcommand*{\bibname}{Bibliography}
%\renewcommand*{\refname}{References}

\providecommand{\acro}{}\renewcommand{\acro}{\textsc}
\providecommand{\defin}{}\renewcommand{\defin}{\textsc}

\newcommand{\xmlelmt}{\texttt}
\newcommand{\xmlattr}{\texttt}
\newcommand{\key}{\textbf}
\newcommand{\translit}{\texttt}

% forced pagebreak
%\newpage

%\usepackage{ulem}
%    \uline{important}   underlined text
%    \uuline{urgent}     double-underlined text
%    \uwave{boat}        wavy underline
%    \sout{wrong}        line drawn through word (cross out, strike out)
%    \xout{removed}      marked over with //////.
%    {\em phasized\/}  | In LaTeX, by default, these are underlined; use
%    \emph{asized}     | \normalem or [normalem] to restore italics
%    \useunder{\uwave}{\bfseries}{\textbf}
%                        use wavy underline in place of bold face

%\usepackage{natbib}
%\usepackage[authoryear]{natbib}
%\citet for "textual" (in text, author's name clear) or
%\citep for "parenthetical" (author's name in parens)
%\citet[before][after]{key} e.g. \citet[see][p.~47]{foo:1997}
%\citet[after]{key}
%\citep  similar
%
%\citet*{key}  list all authors, not just et.al
%\citetext{priv.\ comm.} comes out as (priv. comm.)
%
%just the author or year
%\citeauthor{key} comes out as "Jones et al."
%\citeauthor*{key} comes out as "Jones, Sacco and Vanzetti"
%\citeyear{key}   comes out as 1990
%\citeyearpar{key}            (1990)
%
%Rare stuff:
%use \Citet and \Citep for exceptional forcing of initcap on names
%like 'della Robbia' when it appears first in a sentence.
%
%\citealt like \citet but without parens
%\citealp like \citep but without parens
%


% fancyheadings from The Book (old, obsolete, I think)
%\usepackage{fancyheadings}
%\pagestyle{fancyplain}
% remember the chapter title
%\renewcommand{\chaptermark}[1]{\markboth{#1}{}}
%\renewcommand{\sectionmark}[1]{\markright{\thesection\ #1}}
%\lhead[\fancyplain{}{\small\scshape\thepage}]{\fancyplain{}{\small\scshape\rightmark}}
%\rhead[\fancyplain{}{\small\scshape\leftmark}]{\fancyplain{}{\small\scshape\thepage}}
%\cfoot{}

% new fancyhdr package
%\usepackage{fancyhdr}
%\pagestyle{fancy}
%\fancyhead{}

%% L/C/R denote left/center/right header (or footer) elements
%% E/O denote even/odd pages

%% \leftmark, \rightmark are chapter/section headings generated by the 
%% book document class

%\fancyhead[LE,RO]{\slshape\thepage}
%\fancyhead[RE]{\slshape \leftmark}
%\fancyhead[LO]{\slshape \rightmark}
%\fancyfoot[LO,LE]{\slshape Short Course on Asymptotics}
%\fancyfoot[C]{}
%\fancyfoot[RO,RE]{\slshape 7/15/2002}

% another example
%\fancyhead[LE]{\thepage}
%\fancyhead[CE]{\bfseries Beesley}
%\fancyfoot[CE]{First Draft}
%\fancyhead[CO]{\bfseries My Article Title}
%\fancyhead[RO]{\thepage}
%\fancyfoot[CO]{For Review and Editing Only}
%\renewcommand{\footrulewidth}{0.4pt}




% bigbox -- puts a box around a float
% for {figure}, {table} or {center}

\newdimen\boxfigwidth  % width of figure box

\def\bigbox{\begingroup
  % Figure out how wide to set the box in
  \boxfigwidth=\hsize
  \advance\boxfigwidth by -2\fboxrule
  \advance\boxfigwidth by -2\fboxsep
  \setbox4=\vbox\bgroup\hsize\boxfigwidth
  % Make an invisible hrule so that
  % the box is exactly this wide
  \hrule height0pt width\boxfigwidth\smallskip%
% Some environments like TABBING and other LIST environments
% use this measure of line size -
% \LINEWIDTH=\HSIZE-\LEFTMARGIN-\RIGHTMARGIN?
  \linewidth=\boxfigwidth
}
\def\endbigbox{\smallskip\egroup\fbox{\box4}\endgroup}


% example
% \begin{figure}
%   \begin{bigbox}
%     \begin{whatever}...\end{whatever}
%     \caption{}
%     \label{}
%   \end{bigbox}
% \end{figure}
% 
% N.B. put the caption and label inside the bigbox

%\usepackage{graphicx}
% Sample Graphics inclusion; needs graphicx package
%\begin{figure}[ht]
%\begin{bigbox}
%\centering
%\includegraphics{foobar.pdf}   # e.g. PNG, PDF or JPG, _not_ EPS
%\caption{}
%\label{lab:XXX}
%\end{bigbox}
%\end{figure}

%\pagestyle{empty}  % to suppress page numbering

% turn text upside down
%\reflectbox{\textipa{\textlhookp}}
% prevent line break:   \mbox{...}

\hyphenation{hy-po-cri-tical ri-bald}

%%%%%%%%%%%%%%%%%%%%  title %%%%%%%%%%%%%%%%%%%%%%%%%%%%%%

\title{Using Priority Union to Reduce a Sequence of FST Lookups to a Single
  FST Lookup}
\author{Kenneth R.~Beesley}

% to override automatic "today" date
\date{15 September 2006}

%\usepackage{makeidx}
%\makeindex
% see \printindex below in the document
%\usepackage{showidx}   % print proofs showing indexed locations!!!

%%%%%%%%%%%%%%%%%%%%%% document %%%%%%%%%%%%%%%%%%%%%%%%%%

\begin{document}
%\setcode{buck}
\maketitle

\section{Introduction}

Some current Inxight linguistic products contains sets of finite-state
transducers and an algorithm that consults them in an ordered
sequence, FST1, FST2, FST3, \dots{} FSTn, until a solution is found.
That is, the input word \emph{iword} is looked up first in FST1, and
if an analysis is returned by FST1, the process terminates.  Else if
\emph{iword} is not analyzed by FST1, the algorithm then tries to lookup
\emph{iword} in FST2, and so on, always stopping as soon as a solution
is found.

Intuitively, \emph{iword} is being looked up in a set of
FSTs in which FST1 has priority over FST2, FST2 has priority over
FST3, and so on.  The relative priorities are enforced by maintaining
separate FSTs and apply them one by one in an algorithms implemented
in imperative code.

Finite state technology would appear to offer a simpler and more
efficient alternative to this algorithm.

\section{A Finite State Alternative}

\subsection{Some Preliminaries}

A finite state transducer encodes a regular relation, i.e. a relation
between two regular languages.  In the Xerox/PARC tradition these
languages are visualized as a ``lower'' language consisting of
orthographical strings like \emph{book} and an ``upper'' language
consisting of strings (typically something like  book[Noun][Sg] and book[Verb][Sg],
that can be read as morphological analyses of the orthographical
words, showing the baseform, part of speech, tense, aspect, number,
etc.  In finite state transducer encoding a relation,
each string in the lower language is ``mapped'' to one or more strings
of the upper language; i.e. each string in the lower language has one
or more analyses associated with it.  And each analysis string in the
upper language has one or more (but typically one) 


%\part  only for books
%\chapter for books and reports
\section{}

\subsection{}
\subsubsection{}
\paragraph{}
\subparagraph{}

%\appendix
% causes subsequent sections to be lettered rather than numbered
%\section*{}  % to suppress lettering A, B, C of appendices

% N.B. \begin{thebibliography}..\end{thebibliography} is NOT used with BibTeX

% uncomment one of these when using BibTeX and natbib
%\bibliographystyle{plain}
%\bibliographystile{unsrt}     % list bibliog in order cited in the text
%\bibliographystyle{chicago}
%\bibliographystyle{apalike}
% other possibilities: alpha, abbrv

%\bibliography{foo,bar}     
% load foo.bib, bar,bib, ... give relative paths

%\printindex

\end{document}
