
\documentclass[letterpaper,12pt]{article}
% \documentclass[a4paper,12pt]{article}
% twocolumn letterpaper 10pt 11pt twoside

% for other type sizes, 8, 9, 10, 11, 12, 14pt, 17pt, 20pt
% \documentclass[14pt]{extarticle}
% also extbook, extletter available
% \usepackage{extsizes}

%\usepackage{endnotes}
% then put \theendnotes where you want them

\usepackage{times}
\usepackage{xspace}

\usepackage[outerbars]{changebar}
% supports \cbstart and \cbend, also the changebar environment
% need to run pdflatex multiple times


%\usepackage{alltt}
\usepackage{fancyvrb}  % \begin{Verbatim}[fontsize=\small]
% or [fontsize=\footnotesize]
%\usepackage{upquote}
% affects \verb and verbatim
% to get straight quotes, straight single quote, straight double
% quotes in verbatim environments


%\usepackage{latexsym}  % \LaTeX{} for LaTeX;  \LaTeXe{} for LaTeX2e
%\usepackage{mflogo}    % \MF{}  for METAFONT;  \MP for METAPOST
%\usepackage{url}       % \url{http://www.xrce.xerox.com/people/beesley}
%\usepackage{lscape}    % allows \begin{landscape} ... \end{landscape}

%\usepackage{tipa}
%\include{ipamacros}  % my macros to allow same input for DA and IPA
%\usepackage{desalph}
%\usepackage{arabtex} % see usepackage{buck} and setcode{buck} below
%\usepackage{buck}
%\usepackage{mxedruli}

%\usepackage{epsfig}
%\usepackage{pslatex}  % make whole doc. use postscript fonts

% parallel columns, see also multicol
%\usepackage{parcolumns}
%...
%\begin{parcolumns}[<options>]{3}
%\colchunk{ column 1 text }
%\colchunk{ column 2 text }
%\colchunk{ column 3 text }
%\colplacechunks
%...
%\end{parcolumns}


% for more of these names, see Guide to LaTeX, p. 351
%\providecommand*{\abstractname}{}     % in case the style defines one
%\renewcommand*{\abstractname}{Transcriber notes}
%\renewcommand*{\figurename}{Figure}
%\renewcommand*{\tablename}{Table}
%\renewcommand*{\bibname}{Bibliography}
%\renewcommand*{\refname}{References}

\providecommand{\acro}{}\renewcommand{\acro}{\textsc}
\providecommand{\defin}{}\renewcommand{\defin}{\textsc}

\newcommand{\xmlelmt}{\texttt}
\newcommand{\xmlattr}{\texttt}
\newcommand{\key}{\textbf}
\newcommand{\translit}{\texttt}

% forced pagebreak
%\newpage

%\usepackage{ulem}
%    \uline{important}   underlined text
%    \uuline{urgent}     double-underlined text
%    \uwave{boat}        wavy underline
%    \sout{wrong}        line drawn through word (cross out, strike out)
%    \xout{removed}      marked over with //////.
%    {\em phasized\/}  | In LaTeX, by default, these are underlined; use
%    \emph{asized}     | \normalem or [normalem] to restore italics
%    \useunder{\uwave}{\bfseries}{\textbf}
%                        use wavy underline in place of bold face


%                        \usepackage{natbib}
%\usepackage[authoryear]{natbib}
% compatible with \bibliographystyle{plain}, harvard, apalike, chicago, astron, authordate

%\citet for "textual"   \citet{jon90} ->  Jones et al. (1990)
%\citet[before][after]{key} e.g. \citet[see][p.~47]{jon90} --> 
%         see Jones et al.(1990, chap. 2)
%\citet[chap. 2]{jon90}	    -->    	Jones et al. (1990, chap. 2)
%\citet[after]{key}

%   citep for "parenthetical"
%\citep{jon90}	    -->    	(Jones et al., 1990)
%\citep[chap. 2]{jon90}	    -->    	(Jones et al., 1990, chap. 2)
%\citep[see][]{jon90}	    -->    	(see Jones et al., 1990)
%\citep[see][chap. 2]{jon90}	    -->    	(see Jones et al., 1990, chap. 2)

%\citep for "parenthetical" (author's name in parens)
%\citep  similar
%
%\citet*{key}  list all authors, not just et.al
%\citetext{priv.\ comm.} comes out as (priv. comm.)
%
%just the author or year
%\citeauthor{key} comes out as "Jones et al."
%\citeauthor*{key} comes out as "Jones, Sacco and Vanzetti"
%\citeyear{key}   comes out as 1990
%\citeyearpar{key}            (1990)
%
%Rare stuff:
%use \Citet and \Citep for exceptional forcing of initcap on names
%like 'della Robbia' when it appears first in a sentence.
%
%\citealt like \citet but without parens
%\citealp like \citep but without parens
%


% fancyheadings from The Book (old, obsolete, I think)
%\usepackage{fancyheadings}
%\pagestyle{fancyplain}
% remember the chapter title
%\renewcommand{\chaptermark}[1]{\markboth{#1}{}}
%\renewcommand{\sectionmark}[1]{\markright{\thesection\ #1}}
%\lhead[\fancyplain{}{\small\scshape\thepage}]{\fancyplain{}{\small\scshape\rightmark}}
%\rhead[\fancyplain{}{\small\scshape\leftmark}]{\fancyplain{}{\small\scshape\thepage}}
%\cfoot{}

% new fancyhdr package
%\usepackage{fancyhdr}
%\pagestyle{fancy}
%\fancyhead{}

%% L/C/R denote left/center/right header (or footer) elements
%% E/O denote even/odd pages

%% \leftmark, \rightmark are chapter/section headings generated by the 
%% book document class

%\fancyhead[LE,RO]{\slshape\thepage}
%\fancyhead[RE]{\slshape \leftmark}
%\fancyhead[LO]{\slshape \rightmark}
%\fancyfoot[LO,LE]{\slshape Short Course on Asymptotics}
%\fancyfoot[C]{}
%\fancyfoot[RO,RE]{\slshape 7/15/2002}

% another example
%\fancyhead[LE]{\thepage}
%\fancyhead[CE]{\bfseries Beesley}
%\fancyfoot[CE]{First Draft}
%\fancyhead[CO]{\bfseries My Article Title}
%\fancyhead[RO]{\thepage}
%\fancyfoot[CO]{For Review and Editing Only}
%\renewcommand{\footrulewidth}{0.4pt}

% \vspace{.5cm}
% c, l, r, p{1cm}
%\begin{tabular}{}
%\hline
%   &  &  &   \\
%\hline
%\end{tabular}
% \vspace{.5cm}


% bigbox -- puts a box around a float
% for {figure}, {table} or {center}

\newdimen\boxfigwidth  % width of figure box

\def\bigbox{\begingroup
  % Figure out how wide to set the box in
  \boxfigwidth=\hsize
  \advance\boxfigwidth by -2\fboxrule
  \advance\boxfigwidth by -2\fboxsep
  \setbox4=\vbox\bgroup\hsize\boxfigwidth
  % Make an invisible hrule so that
  % the box is exactly this wide
  \hrule height0pt width\boxfigwidth\smallskip%
% Some environments like TABBING and other LIST environments
% use this measure of line size -
% \LINEWIDTH=\HSIZE-\LEFTMARGIN-\RIGHTMARGIN?
  \linewidth=\boxfigwidth
}
\def\endbigbox{\smallskip\egroup\fbox{\box4}\endgroup}


% example
% \begin{figure}
%   \begin{bigbox}
%     \begin{whatever}...\end{whatever}
%     \caption{}
%     \label{}
%   \end{bigbox}
% \end{figure}
% 
% N.B. put the caption and label inside the bigbox

%\usepackage{graphicx}
% Sample Graphics inclusion; needs graphicx package
%\begin{figure}[ht]
%\begin{bigbox}
%\centering
%\includegraphics{foobar.pdf}   # e.g. PNG, PDF or JPG, _not_ EPS
%\caption{}
%\label{lab:XXX}
%\end{bigbox}
%\end{figure}

%\pagestyle{empty}  % to suppress page numbering

% turn text upside down
%\reflectbox{\textipa{\textlhookp}}
% prevent line break:   \mbox{...}

\hyphenation{hy-po-cri-tical ri-bald}

%%%%%%%%%%%%%%%%%%%%  title %%%%%%%%%%%%%%%%%%%%%%%%%%%%%%

\title{Notes on RTN Support}
\author{Kenneth R.~Beesley}

% to override automatic "today" date
\date{8 July 2010}

%\usepackage{makeidx}
%\makeindex
% see \printindex below in the document
%\usepackage{showidx}   % print proofs showing indexed locations!!!

%%%%%%%%%%%%%%%%%%%%%% document %%%%%%%%%%%%%%%%%%%%%%%%%%

\begin{document}
\maketitle

%\tableofcontents
%\listoffigures
%\listoftables

\begin{abstract}
Thoughts from a talk with Phil Sours, 8 July 2010, on Kleene functions to
support RTN functionality.
\end{abstract}

\section{Syntax for Denoting a Push to a Subnetwork}

\subsection{Phil's Current Approach}

Phil proposed that \verb!'$foo'!, a multichar symbol whose name is the name
of a subnetwork, be used as the Kleene syntax to designate a push to the
\verb!$foo! subnetwork.  The label on the arc would look like \verb!$foo!,
and the runtime code would need to detect it and treat it specially.

In email, I expressed two concerns:

\begin{enumerate}
\item
It imposes a change/restriction on the Kleene language, which normally
allows you to put any characters inside single quotes as the name of a
multichar symbol.  Now any multichar symbol starting with a dollar sign
would have a special meaning.  Users might blunder unintentionally into
creating a multichar symbol with a special meaning.

\item
I fear that the syntax \verb!'$foo'! does not obviously suggest a push to a
subnetwork.
\end{enumerate}

As for allowing any characters to appear inside single quotes as the name
of a multichar symbol, there will almost certainly need to be some
restrictions.  Internally I have often used multichar symbols starting with two
underscores, e.g.\@ \verb!__@#@!, for special purposes, and it would be appropriate to document
that, e.g.\@ ``Names starting with two underscores are reserved for system use.''

\subsection{Ken's Counter-Proposal}

I proposed to implement a special built-in function, \verb!$&push()!, that
takes an argument like \verb!$foo!, retrieves the String image (rather than
the network itself), and perhaps pre-pends two underscores to create the
internal symbol \verb!__$foo!.  Whatever we decide to use as the internal
symbol name, the syntax at the user level would remain stable, e.g.

\begin{Verbatim}[fontsize=\small]
$net = a b c $&push($foo) x y z ;
\end{Verbatim}

\noindent
I suggest that \verb!$&push($foo)! is more externally more
readable/intuitive and internally more flexible than just \verb!'$foo'!.

\section{Problems with OTHER}

Multichar symbol labels representing pushes to subnetworks should not be
treated like other ``normal'' symbols when promoting the sigma/OTHER of one
network relative to another.  This is actually a more general problem with
special multichar symbols.

Phil is also currently using \verb!_SUBNETWORKS_! as a label on an arc
leaving from the start state.  This needs to be reviewed (and perhaps
changed to \verb!__SUBNETWORKS!?

\section{Marking an RTN Network}

When a network is intended to serve as an RTN, i.e.\@ when it contains
labels (however spelled) indicating pushes to subnetworks, then the network
is no longer regular in its semantics and has some new restrictions on the
operations that can be performed.

Such networks need to be marked in some way, and the Kleene interpreter
needs to detect any attempt to use them in illegal operations.

\section{Functions to Return the Sigma of a Network}

Phil has requested a Kleene function, or functions, to return the sigma of
a network.  Possibilities include

\begin{itemize}
\item
\verb!$&sigma($fst)!, which would return a finite-state network, probably
being a network with a non-final start state, a second final state, with a
labeled
arc from the start state to the final state for each symbol in the sigma.
Special symbols might need to be excluded.

\item
\verb!$@&sigma($fst)!, which could return an array of networks, each one
having two states and one arc, labeled with a symbol in the sigma.

\item
\verb!#@&sigma($fst)!, which could return an array of integers, each one
representing the code point value of a symbol in the sigma.

\end{itemize}

\noindent
Arrays and functions returning arrays are not yet implemented.

\section{Generalizations of Implode and Explode}

Phil has requested enhancements or generalizations of \verb!$&implode()!
and \verb!$&explode()! that handle multiple paths.

\section{Function to Return a Network, Given its Name}

To facilitate building networks with attached subnetworks, Phil has
requested a function, something like \verb!$&getNetwork($name)!, that would
take a network representing the name and return the network from the symbol
table.  (Correct as necessary.)

If I understand the request correctly, the sequence

\begin{Verbatim}[fontsize=\small]
$name = \$ f o o ;
$net = a b c $&getNetwork($name) ;
\end{Verbatim}

\noindent
or perhaps

\begin{Verbatim}[fontsize=\small]
$name = \$ f o o ;
$net = a b c $&getNetwork($&implode($name)) ;
\end{Verbatim}

\noindent
would be equivalent to

\begin{Verbatim}[fontsize=\small]
$net = a b c $foo ;
\end{Verbatim}

Some questions:

\begin{enumerate}
\item
Would the argument be imploded?
\item
Would it return a handle to the network in the symbol table, or a copy of
it?
\item
...
\end{enumerate}


\end{document}
