\chapter{Control Characters in Kleene}

\label{app:controlcharacters}

\section{Characters Representing Whitespace}

In Kleene source code, as in Java, Unicode characters specified by their
code point value, such as \textbackslash{}u000D (the \acro{carriage
return}) or \textbackslash{}u000A (the \acro{line feed}), are converted
to the real Unicode character even before the tokenizer starts its work.
So a statement typed as

\begin{Verbatim}[fontsize=\small]
$foo = \u000D ;
// or
$foo = \u000A ;
\end{Verbatim}

\noindent
would look to the Kleene tokenizer/parser like

\begin{Verbatim}[fontsize=\small]
$foo =
;
\end{Verbatim}

\noindent
and is not a valid statement because the right-hand side of the assignment is
empty.  But a statement typed as

\begin{Verbatim}[fontsize=\small]
$foo = a b c \u000D e f ;
\end{Verbatim}

\noindent
would look to the tokenizer/parser like

\begin{Verbatim}[fontsize=\small]
$foo = a b c
e f ;
\end{Verbatim}

\noindent
and would compile successfully and match the string ``abcef''.  The carriage
returns and line feeds (newlines) in these examples are just whitespace and
are ignored by the tokenizer and the parser.

\section{Illegal Characters Inside Multichar-symbol Names and
Double-Quoted Strings}

Anything typed as 

\begin{Verbatim}[fontsize=\small]
$foo = a b '\u000Dxyz' e f ;
\end{Verbatim}

\noindent
would look to the tokenizer/parser like

\begin{Verbatim}[fontsize=\small]
$foo = a b '
xyz' e f ;
\end{Verbatim}

\noindent
which is illegal (an attempt to put a newline inside single quotes) and will
raise an exception.  Similarly, anything typed as

\begin{Verbatim}[fontsize=\small]
$foo = a b "\u000Dxyz" e f ;
\end{Verbatim}

\noindent
would look to the tokenizer/parser like

\begin{Verbatim}[fontsize=\small]
$foo = a b "
xyz" e f ;
\end{Verbatim}

\noindent
and is similarly illegal (an attempt to put a carriage return or newline
inside double quotes).  

You can legally type the special sequences \texttt{\textbackslash{}r}
(carriage return) and \texttt{\textbackslash{}n} (newline, line feed) by
themselves, and in double-quoted strings.

\begin{Verbatim}[fontsize=\small]
// legal statements
$foo = \n ;
$foo = a b \r c ;
$foo = a b "cd\nef" g ;
\end{Verbatim}

\noindent
In these cases, the \textbf{\textbackslash{}n} and
\textbf{\textbackslash{}r} get translated by the Kleene tokenizer into
their code point values, and these values duly appear on arcs in the
resulting networks.  The actual values can be seen by \texttt{draw}ing
the network or just invoking the \texttt{sigma} command, e.g.

\begin{Verbatim}[fontsize=\small]
$foo = \n ;
sigma $foo ;
\end{Verbatim}

You cannot type \textbf{\textbackslash{}n} and \textbf{\textbackslash{}r}
inside single quotes to represent newline and carriage return.  Currently
they can be typed, e.g.

\begin{Verbatim}[fontsize=\small]
$foo = 'a\nb' ;
\end{Verbatim}

\noindent

\noindent
but the backslash gets interpreted as a literal backslash.  

KRB:  This
behavior will be reviewed.

\section{Control Characters and Writing Networks to \acro{xml} 1.0}

Although Kleene \emph{per se} can handle the special characters
\textbf{\textbackslash{}b} (the \acro{backspace}) and
\textbf{\textbackslash{}f} (the \acro{form feed}), any attempt to write
the resulting network to \acro{xml} will cause problems because
\acro{xml} 1.0 simply does not allow such control characters to appear in
\acro{xml} text.  For example

\begin{Verbatim}[fontsize=\small]
$foo = \b ;
writeXml $foo, "fooFile.xml" ;
\end{Verbatim}

\noindent
will result in an \acro{xml}-like file that cannot be read back in
because it is technically invalid, containing the \acro{backspace} symbol
that \acro{xml} 1.0 simply does not allow.  \acro{xml} 1.0 does allow the
\textbf{\textbackslash{}n} (\textbackslash{}u000A),
\textbf{\textbackslash{}r} (\textbackslash{}u000D) and
\textbf{\textbackslash{}t} (\textbackslash{}u0009) characters to appear
in text, and so they do not cause the same problem.\footnote{The
\acro{xml} 1.1 standard, first published in 2004, also accepts
\textbf{\textbackslash{}b} (the \acro{backspace}) and
\textbf{\textbackslash{}f} (the \acro{form feed}) in \acro{xml} text, but
\acro{xml} 1.1 has never been well accepted or supported.}


