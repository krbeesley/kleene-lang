\chapter{Introduction}

\section{What is Kleene?}

\Kleene{} is a programming language in the
tradition of the \init{at\&t}
Lextools \citep{roark+sproat:2007},\footnote{\url{http://serrano.ai.uiuc.edu/catms/}}
the \init{sfst-pl} language
\citep{schmid:2005},\footnote{\url{http://www.ims.uni-stuttgart.de/projekte/gramotron/SOFTWARE/SFST.html}}
The \init{hfst}
software,\footnote{\url{http://www.ling.helsinki.fi/kieliteknologia/tutkimus/hfst/}} and the
Xerox/\acro{parc} finite-state
toolkit \citep{beesley+karttunen:2003};\footnote{\url{http://www.fsmbook.com}} 
all of which provide
higher-level programming formalisms built on top of low-level finite-state
libraries.  \Kleene{} allows programmers to specify weighted
finite-state machines, including acceptors that encode regular
languages and two-level transducers that encode regular relations,
using both regular expressions and right-linear phrase-structure
grammars. The language supports variables and functions, plus
familiar control structures such as \verb!if-elsif-else!
statements and \verb!while!
loops.\footnote{Kleene was first reported at the 2008 conference on
Finite-State Methods in Natural-language Processing \citep{beesley:2009}.}

The Java-language \Kleene{} parser, implemented with JavaCC and
JJTree \citep{copeland:2007},\footnote{\url{https://javacc.dev.java.net}} 
is Unicode-capable and portable.\footnote{Kleene currently runs on Red
Hat Enterprise Linux, OS X, and Windows.}
Successfully parsed statements are reduced to abstract
syntax trees (\init{ast}s); and the Java-language interpreter,
implemented using the visitor design
pattern,\footnote{\url{http://en.wikipedia.org/wiki/Visitor_pattern}} 
interprets the \init{ast}s by calling \CPP{} functions in the
OpenFst finite-state 
library
\citep{allauzen+riley+schalkwyk+skut+mohri:2007}\footnote{A beta version of the 
OpenFst library (\url{http://www.openfst.org})
was first released by Google Labs 11 July 2007, and it continues to be
an active project.  The OpenFst
library, and additional \CPP{} code that wraps the OpenFst functions so that they can
be called from Java, must be recompiled for each platform.} 
via the Java Native Interface
(\init{jni}) \citep{gordon:1998,liang:1999}.  

The \Kleene{} graphical user interface (\acro{gui}), implemented with 
the Java Swing library, 
allows interactive creation, testing and graphic display
of finite-state machines.  The overall application window, based on a Swing JDesktopPane,
encloses a Unicode-capable terminal-like window, into which Kleene statements
can be typed,
and a symbol-table window that displays an icon for each defined
machine.  Right-clicking on an icon triggers a pop-up menu with
commands to test, draw, invert, determinize, draw, delete, 
etc.\@ the associated machine.

\Kleene{} is copyrighted by \init{sap} \init{AG} and released under the Apache License, Version 2, which
is a very liberal license that allows even commercial usage without payment of license fees or
royalties.  The OpenFst Library on which Kleene is based is also released under the same Apache License,
Version 2.  See the Apache site at \url{http://www.apache.org/licenses/} for the full details.

\section{The Kleene Name}

\Kleene{} is named in honor of American mathematician Stephen Cole Kleene
(1909--1994), who, among many accomplishments, investigated the properties of regular sets, including
regular languages, and invented the regular-expression metalanguage, which is the foundation of
\Kleene{}-language syntax.  In the \init{usa}, the name Kleene is
commonly pronounced /kliːn/, like the English word \emph{clean}, 
or /ˈkliːni/, but Kleene himself pronounced it
/ˈkleɪni/.\footnote{\url{http://en.wikipedia.org/wiki/Stephen_Cole_Kleene}}

\section{Possible Applications}

Kleene can build acceptors to be used in spell-checking and similar
applications.  Kleene transducers can be used in spelling correction,
tokenization, phonological modeling, morphological analysis and
generation, speech synthesis and generation, and shallow or ``robust''
parsing.

\section{Design Criteria}

The following requirements and desiderata have guided the design and
implementation of \Kleene{}:

\begin{enumerate}

\item
The \Kleene{} language must be compelling, easy to learn and well documented.  
The syntax and semantics should always aim for maximum
familiarity and ``least astonishment.''

\item
Programmers must be able to run pre-edited scripts or
type statements interactively into a \acro{gui}.

\item
\Kleene{} must allow finite-state machines to be defined using 
both regular expressions and right-linear
phrase-structure grammars. 

\item
The syntax should follow, as far as
possible and appropriate, the familiar syntax of Perl-like regular
expressions.  Non-regular features of Perl regular expressions, 
such as back-references, will be
excluded; and operators will be added to denote weights,
two-level relations, subtraction, complementation and intersection, which are
lacking in Perl-like regular expressions.

\item
The abstraction mechanisms must include variables, built-in functions
and user-defined functions.


\item
The syntax will include rule-like expressions similar in semantics and 
function to the
Xerox/\acro{parc} Replace Rules
\citep{karttunen:1995,karttunen+kempe:1995,karttunen:1996,kempe+karttunen:1996,mohri+sproat:1996} that denote regular relations and
compile into finite-state transducers.  

\item
Unicode must be supported from the beginning, not only in data
strings, but also in \Kleene{} identifiers and operators.

\item
The implementation should be maximally portable.

\item
The implementation should be maximally modular, allowing the writing of
interpreters based on various finite-state libraries that might become
available.
\end{enumerate}

\section{Terminology}

A \emph{regular language} is a set of strings and can be encoded as a finite-state acceptor.  A
two-projection (or two-tape) \emph{regular relation} is a set of ordered string pairs and can be
encoded as a finite-state transducer.  Kleene can be used to build finite-state acceptors
and two-projection finite-state transducers.\footnote{Transducers can theoretically have
two or more projections, but most software implementations of transducers---including the
Xerox/\acro{parc}, \init{at\&t}, \init{sfst} and OpenFst implementations---are limited to
just two projections.}

In some traditions, the terms \emph{finite-state machine} and \emph{finite-state automaton}
(plural: \emph{automata}) are used to
denote acceptors, in contrast to finite-state transducers.  Herein, however, I will avoid the
term automaton, and I use the term \emph{finite-state machine} to encompass both
acceptors and transducers.\footnote{In the Xerox/\acro{parc} tradition, the term
\emph{network} is used as a cover term to include both acceptors and transducers, but this
usage has not become general.}

I dislike acronyms and initialisms, especially as they have proliferated and become
increasingly
ambiguous; but some are well established and almost unavoidable in the field.  I will often use
\emph{\fst{}} to refer to finite-state transducers, and \emph{\fsm{}} to refer to finite-state machines
(encompassing both finite-state acceptors and transducers).
Where it is important to distinguish a finite-state machine as an acceptor, it will be
called an acceptor, without abbreviation.

Complicating the issue is the fact that OpenFst finite-state machines are always two-projection
transducers in their structure; each arc in an OpenFst finite-state machine has an ``input
label'' and an associated ``output label.''  In the OpenFst
implementation of finite-state machines, an acceptor is just a special
case of a transducer, which can also be considered an \emph{identity transducer}, wherein each input label is equal to
its associated output label.  These structure issues, and the terminology, will be
discussed again in more detail below in a section on how finite-state machines are
constructed and visualized in various traditions.


